%%%%%%%%%%%%%%%%%%%%%%%%%%%%%%%%%%%%%%%%%%%%%%%%%%%%%%%%%%%%%%%%%%%%%%%%%%%%%%%


\chapter{APÊNDICE E - Condições para validar Tensores de Aspecto através da escolha de Geradores Válidos}
\label{apendiceVII}

%\todo{Traduzir? Esta parte é mais difícil de ser inserida no texto; na tese, tenho que ter cuidado porque esta parte é um pouco mais detalhada.}

Este apêndice traz algumas ideias simples que foram reunidas a partir de discussões sobre a validade dos tensores de aspecto da teoria proposta por \citeauthoronline{purser/1995} (1995) na nota técnica intitulada ``A Geometrical Approach to the Synthesis of Smooth Anisotropic Covariance Operators for Data Assimilation''. Os tensores de aspecto são definidos através de geradores em um tipo de geometria não-Euclidiana que define o domínio no qual as covariâncias são sintetizadas. Os geradores são entes da geometria definida e algumas condições para a sua validade são pesquisadas. 

\begin{otherlanguage}{english}

\begin{center}
\section*{Conditions to validate Aspect Tensors through the choice of Valid Generators}
\end{center}

\textbf{Abstract:} Here are presented some insights regarding the necessary conditions to choose a valid aspect tensor through the choice of the right (and valid) generators.

Let,

\begin{equation}
  \label{apVII_eq:1}
  \mathbf{gg}^{T} =
  \begin{bmatrix}
  	a & b & c \\
    d & e & f \\
    g & h & i 
  \end{bmatrix}
\end{equation}

be a valid aspect tensor. What is(are) the condition(s) that make it valid?

Following \citeauthoronline{purser/1995} (1995), one condition to validate the generator ($\mathbf{g}$) is:

\begin{equation}
  \label{apVII_eq:2}
	det(\mathbf{gg}^{T})=\pm 1
\end{equation}

Considering the condition \ref{apVII_eq:2}, the determinant of the matrix given is:

\begin{equation}
  \label{apVII_eq:3}
	det(\mathbf{gg}^{T}) = a(ei - fg) + b(gf - di) + c(dh - eg) = 1
\end{equation}

Notice that the factors multiplying the coefficients $a$, $b$ and $c$ are the determinants of the \textit{minors} of the given matrix in \ref{apVII_eq:1}:

\begin{equation}
\label{apVII_eq:4}
\begin{split}
	det(M_{1,1}) = (ei - fg) \\
	det(M_{1,2}) = (gf - di) \\
	det(M_{1,3}) = (dh - eg)
\end{split}
\end{equation}

\begin{theorem}
If $det(M_{1,1})\neq0$, $det(M_{1,2})\neq0$ and $det(M_{1,3})\neq0$, then $det(\mathbf{gg}^{T})=0$.
\end{theorem}

\begin{proof}
Let,
$
M_{1,1} =
\begin{bmatrix}
    e & f \\
    h & i 
\end{bmatrix}
$
then, $det(M_{1,1}) = (ei - fg)$, $\forall e, f, i \in \Re$.
\end{proof}

\begin{theorem}
If the determinant of two minors of $\mathbf{gg}^{T}$ (e.g., $M_{1,1}$ and $M_{1,2}$, $M_{1,1}$ and $M_{1,3}$, $M_{1,2}$ and $M_{1,3}$) are zero and the other is non-zero, then $det(\mathbf{gg}^{T})=\pm 1$, iff:

\begin{itemize}
	\item $a = det(M_{1,1}) = \pm 1$
	\item $ei = \pm 1$ and $fh = 0$, so $a = ei = \pm 1$
	\item $ei = 0$ and $fh = \mp 1$, so $a = fh = \mp 1$
\end{itemize}
\end{theorem}

\begin{proof}
Let,
$
\mathbf{gg}^{T} =
\begin{bmatrix}
	a & b & c \\
    d & e & f \\
    g & h & i 
\end{bmatrix}
$
then, $det(\mathbf{gg}^{T}) = a(ei - fg) + b(gf - di) + c(dh - eg)$.
\end{proof}

Now let the minor,
$
M_{1,1} =
\begin{bmatrix}
    e & f \\
    h & i 
\end{bmatrix}
$
then, $det(M_{1,1}) = ei - fh$ and $det(M_{1,1}) = 0$ iff $e = f = h = i, \forall e, f, h, i \in \Re$.

The same can be verified for the elements of the minor $M_{1,2}$ and $M_{1,3}$.

Now, considering that $det(M_{1,2}) = det(M_{1,3}) = 0$, then $\mathbf{gg}^{T}$ would look like:
$
\mathbf{gg}^{T} =
\begin{bmatrix}
	a & b & c \\
    0 & 0 & 0 \\
    0 & 0 & 0 
\end{bmatrix}
$
with $f=g=d=i=h=e=0$. This is a case where $\mathbf{gg}^{T}$ is a tensor somewhere outside the valid region of the geometric domain. But it is also invalid because of the generators:
$
\mathbf{g} =
\begin{bmatrix}
	1 \\
    0 \\
    0 
\end{bmatrix}
$
and
$
\mathbf{g}^{T} =
\begin{bmatrix}
	1 & b & c 
\end{bmatrix}
$
but $det(M_{1,1}) \neq 1$, then
$
\mathbf{gg}^{T} =
\begin{bmatrix}
	a & b & c \\
    0 & x & y \\
    0 & z & w 
\end{bmatrix}
$
with $det(\mathbf{gg}^{T}) = axw - ayz = a(xw - yz) = a det(M_{1,1})$. In this case, $det(M_{1,1})$ relies only on the determinant of the minor $M_{1,1}$.

To make $det(\mathbf{gg}^{T}) \neq 0$, we only require one minor with non-zero elements, i.e.,
$
\mathbf{gg}^{T} =
\begin{bmatrix}
	a & b & c \\
    0 & x & y \\
    0 & z & w 
\end{bmatrix}
$
with $det = axz - awy = a(xz - wy) = a det(M_{1,1})$;
$
\mathbf{gg}^{T} =
\begin{bmatrix}
	a & b & c \\
    x & 0 & y \\
    z & 0 & w 
\end{bmatrix}
$
with $det = byw - bxz = b(yw - xz) = b det(M_{1,2})$;
$
\mathbf{gg}^{T} =
\begin{bmatrix}
	a & b & c \\
    x & y & 0 \\
    z & w & 0 
\end{bmatrix}
$
with $det = xzc - wyc = c(xz - wy) = a det(M_{1,3}) = a det(M_{1,1})$.

Still, in order to have $\mathbf{gg}^{T}$ as a valid tensor (through valid generators), $det(\mathbf{gg}^{T}) = \pm 1$, i.e.,
\begin{itemize}
\item 
$
\begin{cases}
a \times det(M_{1,1}) = 1 \\
a \times det(M_{1,1}) = -1
\end{cases}
$
iff $a = [det(M_{1,1})]^{-1}$ or $a = -[det(M_{1,1})]^{-1}$ which implies $a = xz = \pm 1$ and $wy = 0$, $a = wy = \mp 1$ and $xz = 0$.
\item
$
\begin{cases}
b \times det(M_{1,2}) = 1 \\
b \times det(M_{1,2}) = -1
\end{cases}
$
iff $b = [det(M_{1,2})]^{-1}$ or $a = -[det(M_{1,2})]^{-1}$ which implies $b = xz = \pm 1$ and $wy = 0$, $b = wy = \mp 1$ and $xz = 0$.
\item
$
\begin{cases}
c \times det(M_{1,3}) = 1 \\
c \times det(M_{1,3}) = -1
\end{cases}
$
iff $c = [det(M_{1,3})]^{-1}$ or $a = -[det(M_{1,3})]^{-1}$ which implies $c = xz = \pm 1$ and $wy = 0$, $c = wy = \mp 1$ and $xz = 0$.
\end{itemize}

For all the cases, at least the main column of the minors must be non-zero. Consequently, any identity matrix should be a valid minor.

\end{otherlanguage}