\chapter{CONCLUSÕES}

A matriz de covariâncias dos erros de previsão é um das mais importantes componentes de um sistema de assimilação de dados. Os últimos desenvolvimentos da área demonstram a preocupação na forma como estes erros são especificados e aplicados dentro dos sistemas de assimilação de dados. Novas técnicas tem surgido, inclusive, possibilitando que a matriz estática estacionária seja combinada com uma matriz evolutiva no tempo, de forma que ambas as informações se complementem e que os erros do dia possam ser contabilizados pelos sistemas ao se calcular a análise dos modelos.

Neste trabalho foram estudados os efeitos da aplicação de uma nova matriz de covariâncias dos erros de previsão em combinação com covariâncias oriundas de um filtro de Kalman por conjunto. Esta combinação foi aplicada em um sistema de assimilação de dados com bases operacionais, e foi executado por um período de dois meses em um experimento cíclico de assimilação de dados. O sistema construído para esta aplicação, corresponde a um sistema híbrido 3DVar. A característica híbrida desse sistema corresponde as combinações lineares possíveis entre as covariâncias consideradas. A estrutura variacional do sistema GSI foi utilizada para permitir que as covariâncias de um filtro de Kalman por conjunto pudessem ser inseridas através de uma extensão da variável de controle variacional. Essa extensão, além de ter a característica de levar os estados do conjunto de previsões para os estados determinísticos, tem também a importante propriedade de localizar as covariâncias atualizadas. 

No Capítulo 1, foi feita a introdução ao problema principal da assimilação de dados, que é a determinação das covariâncias dos erros de previsão. Foram apresentados os fatos científicos que possibilitaram o desenvolvimento da previsão numérica de tempo, bem como dos métodos utilizados atualmente para a assimilação de observações meteorológicas. A motivação principal para a realização deste trabalho foi também apresentada e justificada. O CPTEC é o maior centro de previsão numérica do tempo sobre o Hemisfério Sul. A sua missão é a de prover o país com as melhores previsões numéricas de tempo sobre a América do Sul e sobre a região Tropical. Durante algum tempo, o CPTEC operou um sistema de assimilação de dados utilizando uma matriz de covariâncias estática, proveniente de um outro centro, calculada com base em um outro modelo de previsão numérica. Isto sempre representou uma limitação para o desenvolvimento da assimilação de dados no centro, mas é apenas parte do problema. A medida em que os desenvolvimentos com a assimilação de dados começam a serem impulsionadas, outros aspectos do problema também podem ser abordados. Um exemplo disso, é o controle de qualidade das observações, a qual poderá se beneficiar com previsões numéricas mais consistentes, feitas a partir de uma análise mais consistente e oriunda de um sistema de assimilação de dados também mais consistente. As bases que sustentam o argumento principal desta tese, a de que as covariâncias de um filtro de Kalman por conjunto são importantes fontes de informações sobre as variações do fluxo atmosférico, puderam ser testadas com os experimentos realizados.

No Capítulo 2, inicia-se a descrição das componentes da metodologia. São apresentadas as descrições do modelo de circulação geral e as suas principais opções, o sistema de assimilação de dados GSI e os dados de observação utilizados nos experimentos com assimilação. Os Capítulos 3 e 4, fazem parte da metodologia do trabalho, e foram destacados devido a sua importância. Estes capítulos trazem informações novas acerca da natureza da matriz de covariâncias. Foram descritos em detalhes a sua importância, a metodologia para o seu cálculo, as suas estruturas principais e o aspecto do incremento de análise da matriz calculada. Através de um exemplo simples, foi mostrado que o incremento de análise variacional é proporcional as colunas da matriz de covariâncias. Logo, uma matriz de covariâncias calculada com base nas previsões do próprio modelo, se for bem ajustada, é um passo importante para a determinação de boas análises e, consequentemente, boas previsões. Outras metodologias podem também ser utilizadas para o cálculo de covariâncias e estas podem ser testadas no futuro, embora o método NMC pareça ser o método de escolha da maioria dos centros de previsão numérica de tempo, devido a sua simplicidade. Por outro lado, o cálculo de uma matriz de covariâncias é algo bastante específico e inerente ao sistema de assimilação de dados empregado e também a estrutura de assimilação de dados disponível. Em relação aos sistemas de assimilação de dados por conjunto, neste trabalho foram testados dois tipos de filtro de Kalman por conjunto (EnKF e EnSRF). Foram apresentadas as suas formulações básicas e foram comparadas as formas como ambos calculam as perturbações na atualização das covariâncias.

No Capítulo 4, foi apresentada a estrutura variacional disponível para a incorporação das covariâncias do conjunto e como a sua combinação com as covariâncias estáticas são feitas. O estabelecimento de um ciclo de assimilação de dados, aproveitando a estrutura previamente estabelecida do sistema G3DVAR, permitiu que o sistema híbrido 3DVar pudesse ser exercitado. A principal vantagem desta implementação está no fato de que o sistema, apesar das limitações computacionais, pode ser utilizado em diversos tipos de estudos, pois utiliza um modelo de previsão de tempo operacional em baixa resolução (para o qual já se tem a estrutura necessária para calcular a matriz de covariâncias), observações meteorológicas provenientes de um fluxo contínuo e, portanto, também operacional. A grande limitação, entretanto, está na capacidade computacional que o centro atualmente possui. A resolução escolhida para a realização dos experimentos reflete não apenas os aspectos do conjunto de previsões em si, i.e, um conjunto de previsões com 40 membros (que pode ser considerado um conjunto grande, dependendo da resolução escolhida), mas também das limitações computacionais envolvidas no processo de desenvolvimento deste trabalho. Seria muito importante validar os resultados encontrados para outras épocas do ano e com resoluções maiores, focando mais os sistemas convectivos típicos dos meses de verão sobre a América do Sul.

No Capítulo 5, são apresentados os resultados obtidos com os experimentos propostos com o sistema híbrido 3DVar. Foram realizados 6 experimentos numéricos, sendo 2 deles experimentos controle (com e sem a assimilação de dados 3DVar pura) e 4 deles com a assimilação de dados híbrida 3DVar. Entre estes experimentos, buscou-se variar a quantidade de contribuição das covariâncias do conjunto as covariâncias estáticas. Como um dos experimentos controle foi com a assimilação de dados 3DVar pura, definiu-se que os demais experimentos deveriam conter as contribuições do conjunto em 50\% e 75\%. Não foram realizados experimentos com 100\% de contribuição das covariâncias por motivo de tempo de processamento e espaço disponível para armazenamento. Um único experimento híbrido (para o período de 2 meses na resolução TQ0062L028) ocupa aproximadamente 6TB, sem contar as previsões de 5 dias que foram realizadas na avaliação dos resultados. Além disso, pôde-se verificar que quanto maior a contribuição do conjunto, não necessariamente será melhor o resultado. Esta equação, depende também de outros fatores, como a configuração do próprio filtro de Kalman por conjuntos. A avaliação das análises e previsões do sistema híbrido foram organizadas de forma a avaliar o conjunto de análises por meio de estatísticas de inovação, a qual compreende uma razão entre o desvio padrão das inovações do conjunto em relação a raiz quadrada do espalhamento total. Neste caso, o espalhamento total foi considerado como sendo a soma do espalhamento do conjunto com o erro da observação. Nesta avaliação, encontrou-se que as análises do EnSRF responderam melhor as configurações padrão de inflação e localização das covariâncias. Isto pôde ser verificado através das diferenças entre os \textit{priors} (previsões) e os \textit{posteriors} (as análises). No caso dos conjuntos de análises do EnKF, as curvas representando estas quantidades estiveram muito próximas entre si ao longo do tempo de simulação.

A avaliação dos experimentos foi feita também em relação a habilidade das previsões de até 5 dias em relação as análises dos experimentos. Nesta avaliação, foi empregada a correlação de anomalia junto com um teste t-Student, a fim de se verificar a significância dos resultados. A partir destes resultados, pôde-se perceber que as análises dos sistema híbrido 3DVar foram, para muitas variáveis e, em várias das regiões avaliadas, superiores as análises do 3DVar puro e NCEP. Vale ressaltar que as análises do sistema 3DVar puro também utilizaram a mesma matriz de covariâncias estática que os experimentos com o método híbrido 3DVar. Os principais resultados obtidos nesta avaliação mostram que sobre a região tropical, o experimento EnKF75 na previsão da pressão em superfície apresentou ganho de 12 horas com destreza de 80\%; ganhos expressivos com o experimento EnSRF75 para as previsões da umidade em 925 hPa, em que o experimento 3DVar puro apresenta 80\% de destreza com 96 horas de previsão (o experimento EnSRF75, mantém essa destreza para previsões mais longas); o mesmo desempenho foi observado para as previsões da temperatura em 850 hPa, com destreza de 80\% em 72 horas de previsão com o experimento EnSRF75. Sobre a América do Sul, os destaques desta avaliação referem-se a umidade em 925 hPa, onde o experimento EnSRF75 apresentou o melhor resultado e a temperatura em 850 hPa, com destreza de 80\% para as previsões de 72 horas também com o experimento EnSRF75. Estes resultados mostram ganhos bastante importantes para o modelo de circulação geral do CPTEC, principalmente sobre as variáveis representadas mais próximas a superfície e sobre as regiões de interesse, ou seja, a região tropical e a América do Sul.

Uma avaliação das previsões de 24 horas a partir dos experimentos também foi feita, muito embora seja bastante difícil representar a precipitação convectiva em baixa resolução (TQ0062L028, \textasciitilde{}200 km sobre a região equatorial). Nesta avaliação, foram utilizados os dados do GPCP na avaliação das médias temporais para o mês de Janeiro de 2013 e na avaliação das médias espaciais para o mesmo período. Apesar da boa destreza na previsão das variáveis de estado do modelo (como temperatura, pressão e umidade), a representação da precipitação pelos experimentos não foi muito satisfatória. Vários elementos podem contribuir para o desempenho obtido: escolha das parametrizações \textit{Cumulus}, difusões horizontal e vertical, esquemas de radiação e nuvens. Além disso, pode-se especular também sobre a sensibilidade da física do modelo em relação aos ajustes na matriz de covariâncias. %Em outros estudos com a inclusão de novas fontes de dados de observação, como por exemplo o recém lançado Global Precipitation Measurement (GPM, o projeto que dá continuidade ao TRMM), as amplitudes das variâncias dos erros dos dados de hidrometeoros são incluídos na matriz de covariâncias para que o sistema GSI possa realizar a sua assimilação. Este é uma forma de se representar informações sobre os processos úmidos da atmosfera dentro do contexto da assimilação de dados. De outra forma, esta representação seria indireta através de dados de radiância ou através apenas da inclusão das amplitudes de quantidades relativa na matriz de covariâncias dos erros de observação. Nesta avaliação, apesar da precipitação dos experimentos não terem refletido bem os acumulados de precipitação, destaca-se o experimento EnKF75 cuja média de precipitação sobre a América do Sul foi mais próxima do GPCP.

Além da avaliação das previsões de precipitação em 24 horas, escolheu-se um dos experimentos híbridos 3DVar para se fazer um estudo de caso sobre um episódio das ZCAS, em Janeiro de 2013. O experimento escolhido foi o experimento EnKF75 pelo fato de ter representado, na média espacial, o acumulado de precipitação mais próximo do GPCP. Além deste experimento, decidiu-se também incluir os experimentos REF e 3DVar (puro) para comparação, uma vez que estes experimentos são os controles. Neste estudo de caso, identificaram-se 3 casos de ZCAS, conforme documentado no boletim Climanálise, os quais puderam ser identificados em todos os experimentos realizados. Para isso, utilizou-se como critério de identificação da atividade das ZCAS, a componente zonal do vento em 850 hPa. As diferenças entre os experimentos na identificação, foi em relação a intensidade, duração (início e fim) do escoamento do vento seguindo o critério. O caso escolhido foi o primeiro, entre 9 e 14 de Janeiro, para o qual, foram verificadas as condições dinâmicas representadas pelos experimentos REF, 3DVar e EnKF75. Os experimentos 3DVar e EnKF75 não diferiram muito entre si, uma vez que estes experimentos foram realizados com ciclos de assimilação, utilizando o mesmo conjunto de observações e matriz de covariâncias. Em relação ao experimento REF, este mostrou que as previsões de 48 horas do campo de umidade são subestimadas em relação ao que foi encontrado com os experimentos com assimilação de dados. O experimento EnKF75, por outro lado, mostrou potencial para amenizar a subestimativa do campo de umidade em relação aos demais.

Ainda no escopo desta pesquisa, em estudos preliminares a aplicação das covariâncias de um filtro de Kalman por conjunto e sua combinação com covariâncias estáticas, foram verificadas as diferenças entre as matrizes de covariâncias calculadas para os horários sinóticos individuais. Por exemplo, uma matriz de covariâncias calculada apenas com os pares de previsões de 48 e 24 horas (utilizando-se o método NMC) válidos para as 00, 06, 12 e 18Z. Estas matrizes individuais foram testadas em experimentos de assimilação de dados de duas formas distintas: realizando-se os ciclos de assimilação com cada matriz e realizando-se ciclos de assimilação em que as matrizes de covariâncias são alternadas, cada uma sendo utilizada em seu respectivo horário sinótico. Para esta verificação, encontrou-se que a alternância dos pares de previsões, em relação a matriz completa pode, em alguns casos, permitir que mais observações sejam assimiladas em cada ciclo. Este resultado está relacionado ao fato de que uma matriz de covariâncias própria permite que a análise utilize a informação de referência da superfície a partir da previsão de curto prazo para aplicar o operador observação e calcular o perfil do modelo relativo aos perfis das radiâncias a serem assimiladas. Neste caso, as diferenças na quantidade total de dados assimilados está justamente no aumento de dados de radiâncias assimilados. 

%Nos experimentos realizados, verificou-se a forma da matriz de covariâncias em um experimento de observação única. Este tipo de experimento é importante e deve ser mandatório sempre que algum parâmetro seja modificado dentro do sistema. Esta é uma das únicas formas de se aferir o impacto direto na realização do sistema, frente à quantidade de observações rotineiramente assimiladas. Além disso, foi realizada também uma caracterização das estruturas de comprimentos de escala, variâncias e matrizes de projeção da nova matriz, e foi mostrado em comparação com o uso da matriz atual, que as estruturas associadas ao erro da previsão estão em boa parte localizadas sobre o Hemisfério Sul. Tal fato pode nos levar a entender melhor como a assimilação de dados pode ser calibrada para esta região do globo, onde é notável a menor quantidade de observações de superfície e a quantidade de observações de radiâncias sobre os oceanos. Também, como forma de verificação, este diagnóstico será útil para trabalhos futuros, em que um ajuste fino deve ser realizado.

%Como forma de se verificar como a quantidade de observações assimiladas muda a cada ciclo de assimilação de dados, foi realizado uma série de experimentos em que matrizes exclusivas de cada horário sinótico foram empregadas. Nestes experimentos, encontrou-se que as matrizes de covariâncias calculadas com base no MCGA-CPTEC/INPE, apresentam melhores resultados em comparação com a matriz do NCEP, como esperado. Além disso, foi testado também, um experimento em que as matrizes exclusivas foram alternadas nos experimentos. 

Buscou-se verificar também, o condicionamento das previsões para até 15 dias a partir dos conjuntos de análises realizados com os 40 membros, utilizados nos experimentos híbridos 3DVar. Este experimento não pôde ser trazido para o escopo da tese por duas razões: 1) os resultados de destreza das previsões em termos do ``Continuous Rank Probability Skill Score'' (CRPSS) não foram satisfatórios em relação ao que se já tem com o Sistema de Previsão por Conjuntos (SPCON) do CPTEC, ou seja, não houve vantagem em se realizar o sistema híbrido para esta finalidade (previsão estendida para até 15 dias), na forma como ele está configurado; 2) o sistema híbrido 3DVar não é necessariamente desenhado para previsões até 15 dias, mas para previsões numéricas até 5 ou 7 dias. Isto é consequência da finalidade a que o sistema híbrido 3DVar se presta, ou seja, assimilação de dados. Isso quer dizer que previsões de curto prazo são utilizadas para atualização e então são integradas por um modelo de previsão de tempo configurado para prever com uma determinada destreza para até 5 ou 7 dias e que as perturbações realizadas, mediante as configurações de inflação e localização, não são adequadas para previsões de mais longo prazo. A aplicação de um sistema híbrido para a melhoria do sistema de previsões do conjunto do CPTEC, irá requerer uma reconfiguração do sistema para que este possa ser utilizado, por exemplo, em dupla resolução. Neste caso, uma resolução mais alta poderá ser utilizada para as previsões até 5 dias e utilizando-se a análise híbrida variacional e, a partir disso, uma outra resolução (mais baixa) para as previsões estendidas em conjunto com uma configuração mais adequada do modelo poderá ser utilizada para esta finalidade.

%Um dos principais desafios encontrados no desenvolvimento deste trabalho, foi o aspecto computacional como um todo. Seja no tempo de processamento do ciclo de análises, as limitações com relação à escolha da resolução, seja no espaço em disco. O trabalho com conjuntos de análises ou previsões, requer bastante organização e critério. 

% Finaliza
Este trabalho experimentou uma nova matriz de covariâncias dos erros de previsão, aplicada a uma versão modificada do sistema global de assimilação de dados do CPTEC. A realização deste sistema se apresenta como um resultado positivo para o CPTEC e o desenvolvimento de suas rotinas. Mais testes de validação precisam ser feitos, mas os resultados aqui apresentados são promissores.

Diante dos objetivos iniciais proposto para o desenvolvimento desta tese, fez-se primeiro o seguinte questionamento: \textbf{qual é a contribuição das covariâncias do filtro de Kalman por conjunto na determinação das análises e previsões de um sistema de assimilação de dados global?} 

A resposta para esta questão está nos resultados encontrados em que as análises que utilizaram as covariâncias do filtro de Kalman por conjunto, permitiram que o modelo de circulação geral gerasse previsões para algumas variáveis e em algumas regiões, mais bem representadas do que aquelas que consideram apenas as covariâncias estáticas. Isto mostra que se as correlações espaciais dos erros do modelo são semelhantes as correlações entre diferentes previsões (próximas, assim como preconiza o método NMC), então podemos afirmar também que as correlações entre previsões diferentes, - ou seja, os membros de um mesmo conjunto - são também semelhantes e representativas do que chamamos de erros do dia. Logo, o uso das covariâncias provenientes de um filtro de Kalman por conjunto, ajudaram a melhorar a qualidade das análises e previsões através da representação nas analises, das variações espaço-temporais do fluxo atmosférico.

Em relação aos objetivos específicos elencados para o desenvolvimento da tese, foi possível calcular uma nova matriz de covariâncias, a partir da qual foram identificadas e determinadas as estruturas principais e qual o seu impacto nas previsões do modelo de previsão. O sistema GSI foi habilitado para ler as covariâncias de um filtro de Kalman por conjunto, a partir do qual foi estabelecida uma rotina cíclica para que este sistema pudesse combinar as covariâncias estáticas previamente calculadas e as dinâmicas, derivadas do filtro de Kalman por conjunto, com diferentes porcentagens de contribuições. Os efeitos desta atualização foram investigados nas análises e previsões do sistema. Este trabalho mostrou, do ponto de vista teórico e prático, que a matriz de covariâncias dos erros de previsão é uma componente fundamental e determinante do processo de assimilação de dados. A sua melhor representação dentro do sistema de assimilação de dados, tem um impacto positivo e incrementa a qualidade das análises e consequentemente, das previsões. 

%Durante o tempo em que esteve em operação no CPTEC, a matriz em uso no sistema G3DVAR, foi especificada com base em um outro modelo de PNT (modelo GFS/NCEP) e a sua aplicação durante alguns anos no CPTEC mostrou-se deficiente, não permitindo que - por exemplo, mais observações (principalmente radiâncias) pudessem ser assimiladas a cada novo ciclo de análise. Diversos outros fatores também podem colaborar para este desempenho, mas como pode-se comprovar através da teoria, a matriz de covariâncias tem um papel preponderante.

\section{Sugestões para Trabalhos Futuros}

Durante o desenvolvimento deste trabalho, diversos aspectos do sistema híbrido 3DVar não puderam ser completamente testados e acessados. Estes aspectos já tem sido trabalhos em outros centros operacionais e permitem que outras funcionalidades dos sistema sejam acessadas. Além disso, por questões práticas, não foi possível realizar o sistema em resolução mais alta. Inicialmente tinha-se como objetivo realizar o sistema híbrido para a resolução TQ0299L064 (\textasciitilde{}45 km), com uma nova matriz de covariâncias. Porém, os testes iniciais mostraram que o custo computacional é muito alto.

Dentre os aspectos a serem abordados em trabalhos futuros com o sistema híbrido 3DVar, destacam-se:

\begin{itemize}
    \item Investigar o impacto do procedimento de re-centralização do conjunto de previsões em torno da média do conjunto; 
    \item O uso de dupla resolução para previsões mais longas ou em mais alta resolução (neste caso, realiza-se o conjunto de análises do filtro de Kalman por conjunto com mais membros, mas em menor resolução);
    \item Realizar testes de sensibilidade com outros valores de inflação e localização das covariâncias;
    \item Determinar o tamanho ideal do conjunto de previsões em relação a resolução escolhida em aplicação para previsão de tempo de 5 a 15 dias;
    \item Investigar as relações entre o papel da localização do filtro de Kalman por conjunto e os comprimentos de escala nos incrementos de análise do GSI;
    \item Investigar a manutenção do espalhamento do conjunto de análises dentro do sistema híbrido;
    \item Comparar as análises híbridas 3DVar com um esquema em que a matriz de covariâncias é atualizada a partir de uma média móvel no tempo;
    \item Testar outras metodologias para o cálculo da matriz de covariâncias;
    \item Verificar a sensibilidade da aplicação da matriz de covariâncias mediante a modificações na física e dinâmica do modelo de previsão, a partir do qual serão gerados os pares de previsão;
    \item Validar o sistema híbrido 3DVar para outras épocas do ano, na simulação de fenômenos meteorológicos com estruturas fisicamente dependentes (e.g., frentes frias);
    \item Estabelecer uma técnica mais robusta para a avaliação das inovações do conjunto, como a utilização de um sistema de pontuação (\textit{score});
\end{itemize}