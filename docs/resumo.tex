%%%%%%%%%%%%%%%%%%%%%%%%%%%%%%%%%%%%%%%%%%%%%%%%%%%%%%%%%%%%%%%%%%%%%%%%%%%%%%%%
% RESUMO %% obrigatório

\begin{resumo}

\hypertarget{estilo:resumo}{} 

Esta tese de doutorado é dedicada a estudar a determinação e a aplicação das covariâncias dos erros de previsão, por meio da aplicação de um sistema híbrido 3DVar de assimilação de dados. A característica híbrida deste sistema de assimilação de dados refere-se a forma da matriz de covariâncias. Uma matriz de covariâncias híbrida é, no contexto deste trabalho, uma combinação linear entre uma matriz de covariâncias estática (i.e., fixa no tempo) e covariâncias obtidas a partir de um sistema de filtro de Kalman por conjunto (e.g., \textit{Ensemble Kalman Filter}). A combinação entre estas duas espécies de covariâncias tem o benefício de fazer com a matriz de covariâncias estática seja capaz de enxergar, com certa limitação, as variações diárias do fluxo atmosférico, uma característica antes limitada a sistemas computacionalmente mais custosos e complexos. Atualmente, sistemas híbridos tem sido aplicados em centros operacionais de Previsão Numérica de Tempo sob diferentes metodologias. A metodologia aplicada neste trabalho para adicionar as covariâncias dos erros de um conjunto de previsões a matriz de covariâncias estática, utiliza uma extensão da variável de controle do sistema de assimilação de dados variacional. A partir desta metodologia, foi estabelecido um ciclo de assimilação de dados em que análises e previsões são geradas em uma única resolução (TQ0062L028), gerando previsões para até 5 dias. O sistema foi experimentado por durante dois meses do verão austral de 2013, testando-se diferentes porcentagens de contribuições das covariâncias do conjunto. Os resultados obtidos mostram que com a matriz de covariâncias híbrida, as análises produzidas pelo sistema permitem que o modelo de previsão utilizado desempenhe melhor a previsão de diversas variáveis relacionadas ao aspecto físico e dinâmico da atmosfera.

\palavraschave{%
	\palavrachave{Assimilação de Dados}%
	\palavrachave{Assimilação Variacional Tridimensional}%
	\palavrachave{Filtro de Kalman por Conjuntos}%
	\palavrachave{Assimilação de Dados Híbrida}%
	\palavrachave{Matriz de Covariâncias}%
	\palavrachave{Previsão Numérica de Tempo}%
}
 
\end{resumo}