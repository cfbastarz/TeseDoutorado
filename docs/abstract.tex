%%%%%%%%%%%%%%%%%%%%%%%%%%%%%%%%%%%%%%%%%%%%%%%%%%%%%%%%%%%%%%%%%%%%%%%%%%%%%%%%
% ABSTRACT


\begin{abstract}

\selectlanguage{english}

\hypertarget{estilo:abstract}{} 

This thesis is dedicated to the study of background error covariances by means of the application of a hybrid 3DVar data assimilation system. The hybrid characteristic of this system refers to the combination of a static (i.e., fixed in time) background error covariance matrix and covariances drawn from an ensemble of backgrounds through an ensemble data assimilation system (e.g., Ensemble Kalman Filter). The combination between the two kinds of covariances has the benefit of enabling the static covariances to account for the day-to-day variations of the background flow. Recently, hybrid systems have been in use in several Numerical Weather Prediction Centers under different methodologies. The methodology used in this work in order to add the ensemble covariances to the static part, is through an extension of the control variable. From the application of this methodology, a data assimilation cycle was established at a single resolution (TQ0062L028) with forecasts up to 5 days. The system was exercised during two months of the 2013 austral summer, where different amounts of the ensemble contribution to the static covariance have been experienced. The results shows that with the hybrid background error covariance matrix, the system analysis allowed an improvement in the skill of the numerical weather model in the prediction of several dynamical and physical atmospheric parameters.  

\keywords{%
	\palavrachave{Data Assimilation}%
	\palavrachave{3D Variational Assimilation}%
	\palavrachave{Ensemble Kalman Filter}%
	\palavrachave{Hybrid Data Assimilation}%
	\palavrachave{Covariance Matrix}%
	\palavrachave{Numerical Weather Prediction}%
}

\selectlanguage{portuguese}	%% para os documentos escritos em Português
%\selectlanguage{english}	%% para os documentos escritos em Inglês

\end{abstract}