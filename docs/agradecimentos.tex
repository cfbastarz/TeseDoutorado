%%%%%%%%%%%%%%%%%%%%%%%%%%%%%%%%%%%%%%%%%%%%%%%%%%%%%%%%%%%%%%%%%%%%%%%%%%%%%%%%
% AGRADECIMENTOS

\begin{agradecimentos}

\hypertarget{estilo:agradecimentos}{}

Em primeiro lugar, a Deus pela minha vida e pela família que me proveu.

A meus pais, cuja memória me é cara e é sempre lembrada em toda a minha trajetória. 

O desenvolvimento deste trabalho não seria possível sem a colaboração de muitas pessoas, as quais gostaria de expressar meus sinceros agradecimentos.

Aos meu companheiros de trabalho dentro do Grupo de Desenvolvimentos em Assimilação de Dados, os quais faço questão de nomear: Eduardo Khamis, Fábio Diniz, Gustavo Gonçalves, Helena Barbieri, Lucas Amarante e demais companheiros que passaram pelo grupo e que seguiram outros caminhos. Aos meus amigos João Gerd e Bruna Silveira pela amizade, cumplicidade e constante incentivo. Por diversas vezes me auxiliaram nos vários desenvolvimentos, entendimentos e discussões ao longo de meu trabalho.% Agradeço também à Mariana Pallotta pelo auxílio com o estudo de caso.

Aos funcionários e colaboradores do CPTEC/INPE da área administrativa e do programa de pós-graduação, pelo auxílio na organização e atenção aos prazos e etapas envolvidas durante o desenvolvimento do programa de doutoramento.

A minha família, especialmente minha irmã Beatriz Bastarz, pelo incentivo e atenção dados em diversos momentos. 

A minha Helena Cachanhuk, pela paciência, amor, compreensão, amizade, companheirismo, incentivo, conselhos e encorajamentos. Sem a sua presença em minha vida, eu não saberia para onde caminhar. Obrigado por fazer parte da minha vida e me ajudar a melhorar como pessoa a cada dia.

Ao Silvio Nilo pela confiança em mim depositada e paciência na conclusão deste trabalho. Ao Luiz Sapucci pela confiança, incentivo, interesse e também pelos conselhos em diversos momentos durante o desenvolvimento do trabalho.

Ao meu orientador Dirceu Herdies, pela oportunidade e incentivo. Agradeço por fazer parte de minha formação profissional desde o meu começo no CPTEC como aluno de mestrado e agora, como funcionário do INPE, instituição da qual tenho muito orgulho de fazer parte.

Ao Ricardo Todling do GMAO/NASA pela boa vontade em me receber no GMAO durante os 9 meses em que estive lá. Agradeço pela paciência, orientação, entusiasmo e boa vontade para comigo. Ao Stephen Cohn, também do GMAO/NASA, pela acolhida, paciência e orientação. Foi um grande privilégio poder aprender com um grande matemático, do qual pude observar e absorver características que agora fazem parte do meu pensamento matemático e crítico.

Aos demais colegas com os quais tive contato durante a realização do programa de doutoramento e que direta ou indiretamente me auxiliaram no trabalho. Meus sinceros agradecimentos.

Ao INPE pela concessão da bolsa institucional (via CAPES), no período de 2012 a 2013.

Ao CNPq pela bolsa de doutorado no período de 2013 a 2014 (processo número 140938/2014-1).

À CAPES pela concessão da bolsa para a realização do doutorado sanduíche no exterior, no período de 2014 a 2015 (processo número BEX 99999.008036/2014-04).

\end{agradecimentos}